% !TEX root = ../masterthesis.tex
% !TEX spellcheck = de-DE

\chapter*{Zusammenfassung}
\label{cha:zusammenfassung}


Geräte, die auf Quantentechnologie basieren, sind prädestiniert, Teil unseres täglichen Lebens zu werden.
Halbleiterquantenpunkte mit ihren diskreten elektronischen Zuständen sind vielversprechende Kandidaten für dieses Gebiet und ideal für die Erforschung der Quantenoptik.
Aus diesem Grund werden in dieser Arbeit zunächst Herstellungsverfahren und optische Eigenschaften von GaAs-Quantenpunkten vorgestellt.

Die weitere Diskussion ist in zwei Kapitel unterteilt, deren gemeinsamer Nenner GaAs-Quantenpunkte sind.
Das erste befasst sich mit der resonanten Zwei-Photonen-Anregung mittels "adiabatic rapid passage".
Bei diesem Verfahren werden gechirpte Laserpulse verwendet, deren Charakterisierung, Simulation und Messung den größten Teil dieses Kapitels ausfüllen.
Der zweite Teil behandelt die spektrale Auflösung der Quantenpunktemission mittels eines Scanning Fabry-Pérot-Interferometers.
Hier wird der Leser zunächst durch den Aufbau der mathematischen Beschreibung der Fabry-Pérot-Modi geführt.
Anschließend werden Simulationen vorgestellt, mit deren Hilfe die Komponenten des Interferometers dimensioniert werden.
Schließlich wird die Feinstruktur des Biexcitons im Rahmen eines Proof-of-Principle-Experiments aufgelöst.

