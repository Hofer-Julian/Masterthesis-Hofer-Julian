% !TEX root = ../masterthesis.tex
% !TEX spellcheck = de-DE

\chapter*{Zusammenfassung}
\label{cha:zusammenfassung}


In dieser Arbeit wird die Anregung von Quantenpunkten mittels Adiabatic Rapid Passage und die Spektroskopie der Quantenpunktemission mit einem Scanning Fabry Pérot Interferometer (FPI) diskutiert.
Zunächst wird die Verwendung von GaAs-Quantenpunkten als Quelle für Einzelphotonen und verschränkten Photonenpaare motiviert und Details zu deren Herstellung und optischen Eigenschaften vorgestellt.
Der für die Messungen verwendete optische Aufbau wird skizziert und es werden Methoden wie die Mikro-Photolumineszenz erläutert.
Anschließend wird die Verwendung von Adiabatic Rapid Passage zur Erzeugung verschränkter Photonen motiviert und deren Implementierung mit frequenz-gechirpten Pulsen erläutert.
Die Theorie hinter dem Chirp wird vorgestellt und führt zur Beschreibung einer Anordnung, mit der der Chirp eines Laserstrahls deterministisch eingestellt werden kann.
Der gechirpte Strahl wurde mit einem interferometrischen Autokorrelator gemessen und der numerische Filter MOSAIC wurde verwendet, um den Chirp-Parameter aus diesen Messungen zu extrahieren.

Schließlich werden Scanning FPIs als Werkzeuge zum Auflösen feiner Merkmale der Quantenpunktemission vorgestellt.
Der Leser / die Leserin wird durch die Theorie der Gaußschen Strahlen und die Berechnung allgemeiner FPI-Eigenschaften geführt, die von Resonatorverlusten bis zu ihren Transmissionsspektrum reichen.
Es werden Methoden zur Unterdrückung höherer Gauß-Modi vorgestellt und Simulationen zur Größenbestimmung der Bauteile des FPIs gezeigt.
Abschließend werden Messungen mit schnellen Fotodioden und CCD-Sensoren vorgestellt und damit die Eignung zum Auflösen der Feinstruktur eines GaAs-Quantenpunktes gezeigt.

