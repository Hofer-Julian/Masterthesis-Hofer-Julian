% !TEX root = ../masterthesis.tex
\chapter{Summary and outlook}
The first topic presented in this thesis is entangled photon generation using \ac{ARP} with frequency-chirped-pulses.
The obstacles in quantifying the chirp just with \ac{IAC} were discussed and one possible solution (\ac{MOSAIC}) was presented.
Finally the pulse expander, used for deterministically adjusting the chirp, was introduced and measurements with and without the pulse expander were presented.
However, the Ti:Sa laser used for these experiments later turned out to be heavily chirped to begin with, which made comparisons difficult.
Additionally it had to be sent away for reparation, which blocked further investigations.
The next steps will be to investigate \ac{MOSAIC} under different scenarios and then, if the chirp can be adjusted, attempt to excite the GaAs \acp{QD} via \ac{ARP}.

The second topic presented is the build-up of a scanning \ac{FPI} from scratch.
After setting up the theoretical foundation of electromagnetic waves and different \ac{FPI} setups, simulations were presented in order to determine component parameters suitable for the wanted \ac{FPI} mode properties.
Afterwards, measurements were presented, which involved aligning the \ac{FPI} with a HeNe laser, mode matching and spatial filtering
Finally, measurements of the \ac{FSS} of a GaAs \ac{QD} was presented and evaluated in order to confirm the consistency between the measured and the simulated result for the free spectral range.
The next steps will be to set up the \ac{FPI} in the confocal mode and measure with it the \ac{FSS} as well.
It might be advisable to buy multiple sets of mirrors in order to fulfil different needs regarding spectral width and resolution.
When this is accomplished the scanning \ac{FPI} can be used as "black box", which comes into action whenever fine details of \ac{QD} emission needs to be resolved.