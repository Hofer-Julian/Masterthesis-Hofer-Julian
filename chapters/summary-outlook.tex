% !TEX root = ../masterthesis.tex
\chapter{Summary and outlook}
\label{cha:summary}
This thesis describes the efforts to improve excitation of quantum dots (\ac{QD}s) and to build up a scanning \acl{FPI} which allows to resolve fine details of \ac{QD} emission spectra.
In order to excite the \ac{QD} via adiabatic rapid passage the chirp of the exciting laser beam needs to be deterministically adjustable.
This again requires to build up a setup suitable to manipulate the chirp and a way to determine the chirp parameter $\alpha$.
The laser beam is measured with an interferometric auto correlator.
However, the shape of this measurement depends only weakly on $\alpha$.
Therefore, the numerical filter \ac{MOSAIC} is applied, which resulted in a signal of which the lower envelope depends strongly on $\alpha$.
The simulations show that a fit of the lower envelope of the \ac{MOSAIC} signal allows to determine the chirp parameter.
A pulse expander was built up, with the goal of deterministically adjusting the chirp by adapting the distance between two optical elements.
Measurements with a Ti:Sa laser with and without a pulse expander were conducted, which suggested that the laser signal without the pulse expander is already heavily chirped.
That is why future experiments will investigate different laser models and then compare the measurements with the chirped signal after the pulse expander.
Additionally, the laser emission shape will be described more accurately in future simulations with a frequency comp instead of a simple Gaussian in order to better understand the influence of this approximation.
Afterwards, \acp{QD} will be excited via adiabatic rapid passage.
This topic was not continued, because the Ti:Sa laser had to undergo maintenance for several months which blocked further investigations.

The second topic of this thesis is the build-up of a scanning \ac{FPI}.
Simulations show that common optical components are sufficient to build up an \ac{FPI} which is able to resolve the zero-phonon line of the \ac{QD} emission.
Measurements with planar mirrors show the instability of this arrangement, which is why we switch to planar-concave mirrors.
The \ac{FPI} is aligned with a HeNe laser and fast photodiodes, allowing to adjust the optical components of the \ac{FPI} with direct feedback.
Together with mode matching and spatial filtering, this allows to resolve the fine structure splitting of the excitonic energy levels of the \ac{QD}.
The transmission is determined to be \SI{5}{\percent} and the thermal shift of the \ac{FPI} modes is estimated to be $\approx \SI{66}{\percent}$ of one free spectral range per Kelvin.
Both the transmission and the thermal drift are sufficient for resolving the splitted excitonic emission line. However, the measured finesse of $45$ is too low to resolve near-Fourier-limited zero-phonon-lines.
Also mode-matching and spatial-filtering is time consuming and error prone, which makes them impractical for common measurements.
Therefore, the \ac{FPI} is going to be used with mirrors of higher reflectivity and lower curvature radius and order to gain higher finesse and to use the \ac{FPI} in the confocal mode.
Here, the curvature radius is equal to the mirror distance, which removes the flexibility to change resolution and free spectral range of the \ac{FPI}.
However, the cavity becomes mode-degenerated and therefore liberates from mode matching considerations.
The needed mirrors are already designed and ordered, however it was not possible to test the operation within the duration of this thesis.
