% !TEX root = ../masterthesis.tex
\chapter{Summary and outlook}

This thesis described the efforts to improve excitation of quantum dots (\ac{QD}s) and to build up a scanning \acl{FPI} which allows to resolve fine details of \ac{QD} emission.
In order to excite the \ac{QD} via adiabatic rapid passage the chirp of the exciting laser beam needed to be deterministically adjustable.
This again required to build up a setup suitable to manipulate the chirp and a way to determine the chirp parameter $\alpha$.
The laser beam was measured with an interferometric auto correlator, however the shape of this measurement depends only weakly $\alpha$.
Therefore the numerical filter \ac{MOSAIC} was applied, which resulted in a signal of which the lower envelope depends strongly on $\alpha$.
The simulations showed that a fit of the lower envelope of the \ac{MOSAIC} signal allowed to determine the chirp parameter.
A pulse expander was built up, with the goal of deterministically adjusting the chirp by adapting the distance between two optical elements.
Measurements with a Ti:Sa laser with and without a pulse expander were conducted, which suggested that the laser signal without the pulse expander is already heavily chirped.
That is why, future experiments will investigate different laser models and then compare the measurements with the chirped signal after the pulse expander.
Afterwards, \acp{QD} will be excited via adiabatic rapid passage.

The second topic of this thesis is the build-up of a scanning \ac{FPI}.
Simulations showed that common optical components are sufficient to build up a \ac{FPI} which is able to resolve the zero-phonon line and the phonon sideband of the \ac{QD} emission.
Measurements with planar mirrors showed the instability of this arrangement, leading to a switch to planar-concave mirrors.
The \ac{FPI} was aligned with a HeNe laser and fast photodiodes, allowing to "live" adjusting \ac{FPI} parameters.
Together with mode matching and spatial filtering, this allowed to resolve the fine structure.
The transmission was determined to be \SI{5}{\percent} and the shift of the \ac{FPI} modes was estimated to be $\approx \SI{66}{\percent}$ of one free spectral range per Kelvin temperature drift.
Both the transmission and the thermal drift proved to be sufficient, however the measured finesse of $45$ is too low for our purpose.
Also mode-matching and spatial-filtering is time consuming and error prone, which makes them unusable for common measurements.
Therefore, the \ac{FPI} is going to be used with mirrors of higher reflectivity and lower curvature radius and order to gain higher finesse and to use the \ac{FPI} in the confocal mode.
Here the curvature radius is equals to the mirror distance, which removes the flexibility to change resolution and free spectral range of the \ac{FPI}.
However, the cavity becomes mode-degenerate and therefore liberates from mode matching considerations.