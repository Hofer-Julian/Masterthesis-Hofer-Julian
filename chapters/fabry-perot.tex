% !TEX root = ../masterthesis.tex
\chapter{Scanning Fabry-Pérot Interferometer}

\section{Motivation}

Resolve QD emission line.

\section{Theory}

\subsection{Gaussian Beam}

Dot-Spectra in far field is (TEM$_{00}$).

\subsection{Fabry-Pérot Interferometer}

The Fabry-Pérot interferometer is an optical resonator developed by Charles Fabry and Alfred Pérot.
An incoming light beam will only be transmitted through the resonator consisting of two semi-transparent mirrors if it fulfils the resonance condition.\cite{kaldewey_coherent_2017}

\subsection{Resonator losses}
For the following discussion of the Fabry-Pérot interferometer, a two-mirror-resonator with the reflecting surfaces facing each other and air as medium in between is assumed. The time the light needs for one roundtrip is then given by ~\cite{ismail_fabry-perot_2016}
\begin{equation}
t_{RT} = \frac{2l}{c}
\end{equation}
where $l$ is the geometrical length of the resonator and $c$ is the speed of light in air. 

The photon-decay time $\tau_c{\nu}$ of the interferometer is then given by
\begin{equation}
\frac{1}{\tau_c} = - \frac{\ln(R_1 \cdot R_2)}{t_{RT}}
\end{equation}
where $R_1$ and $R_2$ are the corresponding intensity reflectivities of the mirrors.

The number of photons at frequency $\nu$ inside the resonator is described by the differential rate equation
\begin{equation}
\frac{d}{dt} \varphi(t) = - \frac{1}{\tau_c}\varphi(t).
\end{equation}
With a number $\varphi_s$ of photons at $t=0$ the integration gives
\begin{equation}
\varphi(t)=\varphi_s e^{-t/\tau_c}
\end{equation}

\subsection{Resonance frequencies and free spectral range}
The round-trip phase shift at frequency $\nu$ is given by 
\begin{equation}
\label{eq:round-trip-phase-shift}
2 \phi(\nu) = 2 \pi \nu t_{RT} = 2 \pi \nu \frac{2l}{c}
\end{equation}
where $\phi(\nu)$ is the single-pass phase shift between the mirrors.

Resonances are visible for frequencies $\nu$ at which the light interferes constructively after one round trip.
Two adjacent resonance frequencies differ in their round trip phase shift by $2 \pi$.
Hence, the free spectral range $\Delta \nu_{FSR}$, the frequency difference between two adjacent resonance frequencies, can be calculated from equation~\eqref{eq:round-trip-phase-shift}
\begin{align}
2\Delta\phi = 2\pi \\
\Rightarrow 2\pi\Delta\nu_{FSR}\frac{2l}{c} = 2\pi\\
\Rightarrow \Delta\nu_{FSR} = \frac{c}{2l}
\end{align}

\subsection{Airy distribution of the Fabry-Pérot interferometer}

\begin{figure}[H]
	\centering
	\includegraphics[width=0.7\linewidth]{figures/fabry-perot/Schematic_of_the_Fabry-Perot_interferometer}
	\caption[Fabry-Pérot interferometer with electric field mirror reflectivities $r_1$ and $r_2$]{Fabry-Pérot interferometer with electric field mirror reflectivities $r_1$ and $r_2$. Indicated in this figure are the electric fields resulting from an incoming $E_{inc}$, the reflected field $E_{refl,1}$ and transmitted field $E_{laun}$. $E_{circ}$ and $E_{circ,b}$ circulate inside the resonator, resulting in $E_{RT}$ after one round-trip. $E_{back}$ is the backwards transmitted field.\cite{noauthor_fabryperot_nodate}}
	\label{fig:schematicofthefabry-perotinterferometer}
\end{figure}
The response of the Fabry-Pérot interferometer is calculated with the circulating-field approach~\cite{ismail_fabry-perot_2016}, where a steady-state is assumed.
$E_{circ}$ is the result of $E_{laun}$ interfering with $E_{RT}$.
$E_{laun}$ is the transmission of the incoming light $E_{inc}$ and $E_{RT}$ is $E_circ$ after one round-trip in the resonator, i.e., after the outcoupling losses of mirror 1 and 2.
Therefore, the field $E_{circ}$ can be calculated from $E_{launch}$ by
\begin{equation}
E_{circ} = E_{laun} + E_{RT} = E_{laun} + r_1 r_2 e^{-i 2 \phi} E_{circ} \Rightarrow \frac{E_{circ}}{E_{laun}} = \frac{1}{1 - r_1 r_2 e^{-i 2 \phi}}
\end{equation}
where $r_1$ and $r_2$ are the electric-field reflectivities of mirror 1 and 2.

The generic Airy distribution considers only light inside the mirrors and is defined as
\begin{equation}
A_{circ} = \frac{I_{circ}}{I_{laun}} = \frac{|E_{circ}|^2}{|E_{laun}|^2} = \frac{1}{\left|1 - r_1 r_2 e^{-i2\phi}\right|^2} = \frac{1}{\left(1-\sqrt{R_1 R_2}\right)^2 + 4\sqrt{R_1 R_2} \sin^2(\phi)}
\end{equation}
with
\begin{align*}
\left|1-r_1 r_2 e^{-i2\phi}\right|^2 &= \left|1- r_1 r_2 \cos(2\phi) + i r_1 r_2 \sin(2\phi)\right|^2 = \left[1-r_1 r_2 \cos(2\phi)\right]^2 + r_1^2 r_2^2 \sin^2(2\phi) \\
 &=1+R_1 R_2 - 2\sqrt{R_1 R_2} \cos(2\phi) = \left(1-\sqrt{R_1 R_2}\right)^2 + 4 \sqrt{R_1 R_2} \sin^2(\phi)
\end{align*}
and additionally $R_i = r_i^2$ and $\cos(2\phi) = 1 -2\sin^2(\phi)$.

\subsection{Simulation}

\begin{figure}[H]
	\centering
	\includegraphics[width=\linewidth]{figures/plots/fabry-perot/simulation-comparison-dot-fabry-perot-modes}
	\caption{}
	\label{fig:simulation-comparison-dot-fabry-perot-modes}
\end{figure}

\section{Setup}

\subsection{Flat mirrors}


\subsection{Concave mirrors}

\subsection{Confocal setup}

\section{Measurements and Results}

