% !TEX root = ../masterthesis.tex
\chapter{Quantum Dot}

\section{Processing}

\section{Properties of our dots}

\subsection{Calculate spectral range of zero-phonon line}
A typical lifetime of a GaAs quantum dot is $\Delta t = 250~ps$.
According to the time-energy uncertainty relation
\begin{align}
\Delta E \cdot \Delta t = \frac{h}{2 \pi}\\
\Rightarrow \Delta E = \SI{2.64}{\micro \electronvolt}
\end{align}
The frequency uncertainty can be obtained through
\begin{equation}
\Delta \nu = \frac{\Delta E}{h}
\end{equation}
By developing $\lambda$into a taylor series
\begin{align}
\label{eq:planck-einstein}
\lambda &= \frac{c}{\nu}\\
\Rightarrow \lambda(\nu) &\approx \lambda(\nu_0) + \lambda'(\nu_0) \cdot (\nu - \nu_0)
\end{align}
$\Delta \lambda$ can be expressed as
\begin{align}
\Delta \lambda &= \lambda(\nu_0 - \Delta \nu) - \lambda(\nu_0)\\
 &= \lambda(\nu_0) - \lambda'(\nu_0)\cdot\Delta \nu - \lambda(\nu_0)\\
 &= - \lambda'(\nu_0)\cdot \Delta \nu.
\end{align}
With equation~\eqref{eq:planck-einstein} this gives
\begin{align}
\Rightarrow \Delta \lambda &= \frac{c}{\nu_0^2} \cdot \Delta \nu = \frac{\lambda_0^2}{c}\cdot\Delta \nu\\
&\approx \SI{1.0}{\pico \metre}
\end{align}

\begin{table}[H]
	\caption[Paramters of GaAs quantum dots used in the laboratory of semiconductor physics department in Linz.]{Parameters of GaAs quantum dots used in the laboratory of semiconductor physics department in Linz.
	Zero-phonon line calculates from from the theoretical limit according to the life time of the excitonic state (as can be seen in equation~\eqref{???}) up to broader lines which are still valued enough to be measured.
	The phonon sideband resembles data taken from \textcite{scholl_resonance_2019}.}
	\label{tab:quantum-dot-emission}
	\begin{tabular}{@{}llll@{}}
		\toprule
		Quantum dot emission & Center wavelength $\lambda_0$           & Spectral range $\Delta \lambda$ & Waveform                  \\ \midrule
		Zero-phonon line               & \SIrange{700}{800}{\nano \metre} & \SIrange{1.0}{1.4}{\pico \metre} & Cauchy\\
		Phonon sideband       & ~\SI{0.25}{\nano \metre} higher than zero-phonon line  & ~\SI{500}{\pico \metre} & Gauss  \\ \bottomrule
	\end{tabular}
\end{table}

\begin{figure}[H]
	\centering
	\includegraphics[width=\linewidth]{figures/fabry-perot/plots/quantum_dot_emission_wavelength_energy}
	\caption[Simulated exciton emission of a GaAs quantum dot]{Simulated exciton emission of a GaAs quantum dot plotted dependant on the wavelength $\lambda$ and the Energy $E$.
	The parameters can be found in table~\ref{tab:quantum-dot-emission}.}
	\label{fig:quantumdotemissionwavelengthenergy}
\end{figure}


Dot-Spectra in far field is (TEM$_{00}$).


\section{Adiabatic Rapid Passage}

