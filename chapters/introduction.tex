% !TEX root = ../masterthesis.tex
\chapter{Introduction}

In the past century, technology was transformed by the first quantum revolution. Scientists and engineers all around the world utilized certain features of quantum mechanics such as energy quantization and wave-particle duality to create devices which are a nowadays a fixed part of everyone's life~\cite{aharonovich_solid-state_2016}.
These technologies, ranging from semiconductor devices to LEDs and lasers, became high-performance components which drive modern information processing and global communication networks.
The second quantum revolution to come will make use of superposition and entanglement~\cite{macfarlane_quantum_2003}.

Entanglement is already the basis of many experiments in quantum computation and its distinct properties can be used for quantum cryptography protocols as well.~\cite{zeilinger_light_2017}.
It describes a physical phenomenon involving two or more particles which quantum states cannot be described independently from each other.
Entanglement is especially useful when the entangled particles are photons, as these can be transported over long distances with optical fibres and are therefore ideal to build up quantum networks~\cite{gisin_quantum_2002}.
Quantum key distribution \acs{QKD} \acused{QKD} under the E$91$ protocol can use the quantum network to exchange entangled photons and then generate a shared encryption key.
Since the photons are entangled, the key can be generated in a way which prevents that potential eavesdropping could go unnoticed.
The key is then used to encrypt the data by one party, the encrypted data is transmitted via classical channels and afterwards it is decrypted with the very same key by the other party.
The E$91$ protocol with its use of entanglement has the advantage that it allows to detect eavesdropping even with lossy channels, which is a guarantee other implementations of \ac{QKD} cannot necessarily fulfil~\cite{ekert_quantum_1991}.

In order to use entangled photons for \ac{QKD} several problems have to be solved first.
Fibre channels suffer from de-coherence and optical losses, degrading the quantum information to a point where they are unusable for quantum protocols and therefore limiting the maximal communication distance.
Usually classical amplifiers would be used to counteract transmission losses, however this does not work with quantum states.
In order to solve this problems, multiple teams are in the process of developing quantum repeaters~\cite{reindl_all-photonic_2018}\cite{duan_long-distance_2001}\cite{simon_quantum_2007}.
Quantum repeaters use quantum teleportation by performing a Bell state measurement on the photon to transmit and a photon entangled to another photon at the next repeater station~\cite{bennett_teleporting_1993}.
The \ac{BSM} teleports the quantum state to the photon at the next repeater station, while destroying the quantum state of the two measured photons in accordance with the no cloning theorem~\cite{park_concept_1970}.
Therefore, the transmission distance can be broken up into segments where the losses are acceptable.

Additionally, the E$91$ protocol requires deterministic sources of single entangled photons.
Two prominent sources are parametric down conversion crystals~\cite{shih_new_1988} and single atoms~\cite{aspect_experimental_1981}.
Sources based on parametric down conversion are currently the brightest ones, however their emission is probabilistic, which means that the number of entangled photon pairs emitted per excitation pulse is statistically distributed.
This can be overcome by filtering, but this solution has the disadvantage of reducing the throughput of the emitter.
Single atoms offer sharp electronic transitions, are free from charge fluctuations and are energetically isolated.
Nowadays advanced techniques~\cite{kuhn_deterministic_2002} are used in order to operate with single-atom-based sources, but their dynamics are relatively low, leading to slow operation rates.

Quantum dots provide an alternative source of single entangled photons.
Quantum dot sources are deterministic, emit maximal one photon pair per excitation and are scalable to nanosized dimensions.
With this motivation in mind, droplet-etched GasAs quantum dots as potential sources are investigated in chapter~\ref{chapter:quantum-dot}, as they are quasi strain-free, of high symmetry and exhibit low values of \ac{FSS}
Entanglement fidelity is limited by re-excitation of photons at the exciton level to the biexciton level before they can decay to the ground state, but this effect can be avoided by resonant two-photon excitation.
However, resonant two-photon excitation usually requires precise control of the intensity of the exciting field in order to inverse the quantum dot from the ground state to the biexciton state~\cite{jayakumar_deterministic_2013}.
Adiabitic rapid passage with frequency-chirped pulses does not suffer from this requirement, which motivates its discussion in chapter~\ref{cha:chirp}.

When it then comes to characterizing \ac{QD} emission, more obstacles arise as fine features of the emission spectrum are not resolvable with a CCD-based spectrometer alone.
A Fabry-Pérot interferometer can be used to resolve \ac{QD} emission while still using the same spectrometer.
It transmits signals of certain frequencies and these frequencies can be adjusted in order to scan through the ranges of interest.
\ac{QD} emission is sent through the scanning interferometer and its output is then be recorded with the CCD allowing to reassemble the complete spectrum.
Chapter~\ref{chapter:scanning-fabry-perot} describes the efforts to build up a scanning Fabry-Pérot interferometer to do exactly that.


