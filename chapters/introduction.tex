% !TEX root = ../masterthesis.tex
\chapter{Introduction}

In the past century, technology was transformed by the first quantum revolution. Scientists and engineers all around the world utilized certain features of quantum mechanics such as energy quantization and wave-particle duality to create devices which are nowadays a fixed part of everyone's life.
These technologies, ranging from semiconductor devices to LEDs and lasers, became high-performance components which drive modern information processing and global communication networks.
The second quantum revolution to come will make use of superposition and entanglement~\cite{aharonovich_solid-state_2016}\cite{macfarlane_quantum_2003}.

Entanglement is already the basis of many experiments in quantum computation and its distinct properties can be used for quantum cryptography protocols as well~\cite{zeilinger_light_2017}.
It describes a physical phenomenon involving two or more particles which quantum states cannot be described independently from each other.
Entanglement is especially useful when the entangled particles are photons, as these can be transported over long distances with optical fibres and are therefore ideal to build up quantum networks~\cite{gisin_quantum_2002}.
Quantum key distribution \acs{QKD} \acused{QKD} under the E$91$ protocol~\cite{ekert_quantum_1991} can use the quantum network to exchange entangled photons and then generate a shared encryption key.
Since the photons are entangled, the key can be generated in a way which prevents that potential eavesdropping could go unnoticed.
The key is then used to encrypt the data by one party, the encrypted data is transmitted via classical channels and afterwards it is decrypted with the very same key by the other party.
The E$91$ protocol with its use of entanglement has the advantage that it allows to detect eavesdropping even with lossy channels, which is a guarantee other implementations of \ac{QKD} cannot necessarily fulfil.

In order to use entangled photons for \ac{QKD} several problems have to be solved first.
Fibre and free space channels suffer from decoherence and optical losses~\cite{sangouard_quantum_2011}.
This reduces the amount of photons to a point where their are unsuitable for secure quantum protocols and therefore limits the maximal communication distance.
Usually classical amplifiers would be used to counteract transmission losses, however this does not work with quantum states because of the no cloning theorem~\cite{park_concept_1970}.
In order to solve this problems, multiple teams are in the process of developing quantum repeaters~\cite{reindl_all-photonic_2018}\cite{duan_long-distance_2001}\cite{simon_quantum_2007}.

The E$91$ protocol and quantum repeaters both require deterministic sources of single entangled photons.
Two prominent sources are parametric down conversion crystals~\cite{shih_new_1988} and single atoms~\cite{norden_entangled_2018}\cite{aspect_experimental_1981}.
However the number of entangled photon pairs with parametric down conversion is probabilistic, which means that the number of entangled photon pairs emitted per excitation pulse is statistically distributed.
Single atoms offer sharp electronic transitions, are free from charge fluctuations and are energetically isolated.
Nowadays advanced techniques~\cite{kuhn_deterministic_2002} are used in order to operate with single-atom-based sources, but their arrangements are bulky and their dynamics are relatively low, leading to slow operation rates.

Quantum dots (\ac{QD} \acused{QD}) provide an alternative source of single entangled photons.
\ac{QD} sources are in principle deterministic, with approximately \SI{90}{\percent} state preparation probability and a current maximum of \SI{85}{\percent} extraction efficiency~\cite{liu_solid-state_2019}.
Under several conditions, they also emit maximally one photon pair per excitation~\cite{schweickert_-demand_2018}.
With this motivation in mind, droplet-etched GasAs quantum dots as potential sources are investigated in chapter~\ref{chapter:quantum-dot}, as they are quasi strain-free, of high symmetry and exhibit low values of \ac{FSS}.
Optical experiments on \acp{QD} require reoccurring methods which are introduced in chapter~\ref{cha:methods}. 
Usually, entangled photon pairs are obtained by exciting the biexciton state via resonant two-photon excitation .
However, resonant two-photon excitation usually requires precise control of the intensity of the exciting field in order to inverse the quantum dot from the ground state to the biexciton state~\cite{jayakumar_deterministic_2013}.
Adiabatic rapid passage and alternative excitation schemes using frequency-chirped pulses do not suffer from this requirement, which motivates its discussion in chapter~\ref{cha:chirp}.

Quantum repeater rely on indistinguishable photons, which means that they match in all possible parameters and degrees of freedom.
Indistinguishability could be estimated by analysing the averaged line shape~\cite{reindl_all-photonic_2018}, however fine features of the emission spectrum are not resolvable with a standard CCD-based spectrometer alone.
A Fabry-Pérot interferometer can be used to resolve \ac{QD} emission spectra while still using the same spectrometer as an accurate power meter.
It transmits signals of certain frequencies and these frequencies can be adjusted in order to scan through the ranges of interest.
\ac{QD} emission is sent through the scanning interferometer and its output power is then be recorded with the CCD allowing to reassemble the complete spectrum.
Chapter~\ref{chapter:scanning-fabry-perot} describes the efforts to build up a scanning Fabry-Pérot interferometer to do exactly that.
Chapter~\ref{cha:summary} summarises the work discussed in this thesis and gives a brief outlook of the next steps.

