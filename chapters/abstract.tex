% !TEX root = ../masterthesis.tex


\chapter*{Abstract}
\label{cha:abstract}

Devices based on quantum technology is predestined to be part of our everyday lifes and semiconductor quantum dots with their bound, discrete electronic states are ideal for the research in quantum optics.
This thesis starts by introducing the properties of droplet-etched GaAs quantum dots which are used in the following discussion. 

The further work is splitted in two chapters with GaAs quantum dots as their common denominator.
The first one deals with resonant two-photon exciation via adiabatic rapid passage.
In this procedure chirped laser pulses and their characterization, simulation and measurements will fill the majority of this chapter.
The second one investigates the capabilities of resolving quantum dot emission with a scanning Fabry Pérot interferometer.
The reader is guided through the build-up of mathematical description of the Fabry Pérot modes.
Afterwards, simulations will be presented in order to size the interferometer.
Finally the fine-structure of a biexciton will be resolved in a proof-of-principle experiment.

