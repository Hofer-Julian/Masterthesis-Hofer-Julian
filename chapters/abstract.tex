% !TEX root = ../masterthesis.tex


\chapter*{Abstract}
\label{cha:abstract}


In this thesis excitation of quantum dots via adiabatic rapid passage and spectroscopy of quantum dot emission with a scanning Fabry Pérot interferometer (FPI) are discussed.
First, the use of GaAs quantum dots as sources of (entangled) single photons is motivated and details of their fabrication and optical properties are presented.
The optical setup used for the measurements is sketched and methods like micro photo-luminescence are explained.
Subsequently, the use of adiabatic rapid passage for entangled photon generation is motivated and its implementation with frequency-chirped pulses is explained.
The theory behind the chirp is presented, leading to the description of a setup used to deterministically adjust the chirp of a laser beam.
The chirped beam was measured with an interferometric autocorrelator and the numerical filter MOSAIC was used to extract the chirp parameter from these measurements.

Finally, scanning FPIs as tools to resolve fine features of quantum dot emission are presented.
The reader is guided through the theory of Gaussian beams and the calculation of common FPI properties, ranging from resonator losses to its transmission spectrum.
Methods to suppress higher Gauss modes are presented and simulations are shown used to size the FPI.
Finally, measurements with fast photodiodes and CCD sensors are presented and with them the suitability for resolving the fine structure of a GaAs quantum dot is shown.   

