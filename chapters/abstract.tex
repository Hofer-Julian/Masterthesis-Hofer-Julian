% !TEX root = ../masterthesis.tex


\chapter*{Abstract}
\label{cha:abstract}


Devices based on quantum technology are predestined to be part of our everyday lifes.
Semiconductor quantum dots with their discrete electronic states are promising candidates for this field and are ideal for research in quantum optics.
That is why this thesis starts by introducing fabrication methods and optical properties of droplet-etched GaAs quantum dots.  

The further work is splitted in two chapters with GaAs quantum dots as their common denominator.
The first one deals with resonant two-photon exciation via adiabatic rapid passage.
In this procedure chirped laser pulses are used and their characterization, simulation and measurements will fill the majority of this chapter.
The second one investigates the capabilities of resolving quantum dot emission with a scanning Fabry Pérot interferometer.
Initially, the reader is guided through the build-up of the mathematical description of the Fabry Pérot modes.
Afterwards, simulations will be presented which are used to size the interferometer components.
Finally the fine-structure of the biexciton will be resolved in a proof-of-principle experiment.

